\documentclass[12pt]{article}
\usepackage[a4paper, margin=1in]{geometry}
\usepackage{graphicx}
\usepackage[serbian]{babel}
\usepackage[T2A]{fontenc}
\usepackage[utf8]{inputenc}

\begin{document}

\bibliographystyle{plain} 


\begin{center}
    \large\textbf{Математички факултет} \\
    \large\textbf{Универзитет у Београду}
    
    \vspace{3cm}
    
    \Large\textbf{Семинарски рад} \\
    \large У оквиру курса Истраживање података 2
    
    \vspace{3cm}
    
    \LARGE\textbf{Тема:} \\
    \Large Истраживање образаца у патогеним острвима секвенци \textit{Escherichia coli} и \textit{Helicobacter pylori}
    
    \vspace{4cm}
    
    \begin{minipage}{0.4\textwidth}
        \begin{flushleft}
            \large\textbf{Професор:} \\
            \large Ненад Митић
        \end{flushleft}
    \end{minipage}
    \begin{minipage}{0.4\textwidth}
        \begin{flushright}
            \large\textbf{Студенти:} \\
            \large Вељко Продан, 163/2019 \\
            \large Маја Миленковић, 160/2019
        \end{flushright}
    \end{minipage}

    \vfill
    
    \large\textbf{Мај 2024}
    
\end{center}

\newpage

\tableofcontents

\addto\captionsserbian{
    \renewcommand\contentsname{Садржај}
}

\newpage

\section{Увод}

\subsection{Патогена острва}

Патогена острва (\textit{Pathogenicity island} - PAI) су посебни генетски елементи присутни на хромозомима великог броја бактеријских патогена. PAI кодирају различите факторе вируленције и обично су одсутни код непатогених сојева исте или блиско сродних врста. Сматрају се подкласом геномских острва која се стичу хоризонталним преносом гена путем трансдукције, коњугације и трансформације, и пружају "квантне скокове" у микробној еволуцији, што доприноси способности микроорганизама да еволуирају \cite{Gal-Mor2006-dv}. 

Једна бактеријска врста може имати више од једног PAI. PAI су кластери гена уграђени у геном патогених организама, хромозомално или екстрахромозомално. Као тип мобилног генетичког елемента, PAI могу варирати од 10-200 кб. Они носе гене који им омогућавају да производе различите факторе вируленције, укључујући адхезине, токсине, системе секреције и друге елементе који им помажу да се вежу за ћелије домаћина, избегну имуни систем и изазову болест \cite{Schmidt2004-xj}. Подаци засновани на бројним секвенцираним бактеријским геномима показују да су PAI присутни у широком спектру грам-позитивних и грам-негативних бактеријских патогена људи, животиња и биљака. Недавна истраживања усмерена на PAI довела су до идентификације многих нових фактора вируленције које ове врсте користе током инфекције својих домаћина \cite{Gal-Mor2006-dv}.

\subsection{Ешерихија коли}

Ешерихија коли (Escherichia coli) је разноврсна бактеријска врста која обухвата како безопасне коменсалне сојеве, тако и патогене сојеве који се налазе у гастроинтестиналном тракту људи и топлокрвних животиња. Патогеност Ешерихије коли, која се постиже хоризонталним трансфером гена који одређују факторе вируленције, омогућава овој бактерији да постане врло разноврстан и адаптибилан патоген одговоран за цревне или ванцревне болести код људи и животиња. Сходно томе, сојеви Е. коли могу се класификовати у три групе: коменсалне/пробиотске сојеве, интестинално патогене сојеве и екстраинтестинално патогене сојеве.

Растућа количина информација о ДНК секвенцама, генерисаних у "ери геномике", помогла је у повећању разумевања фактора и механизама укључених у диверзификацију ове бактеријске врсте \cite{Desvaux2020-mx}. 

\subsection{Хеликобактер пилори}

Хеликобактер пилори је грам-негативни патоген спиралног облика који колонизује антрум и корпус желуца. У последњој деценији, идентификовани су бројни фактори вируленције. Ови елементи омогућавају бактерији да преживи у изузетно киселој средини гастроинтестиналног тракта, доспе до неутралније средине слузног слоја и одупре се имунолошком одговору човека, што резултира перзистенцијом \cite{2022helicobacter}.

Сојеви Хеликобактер пилори показују висок степен генетске хетерогености због геномских преуређења, тачкастих мутација, убацивања и/или брисања гена. Генетички јединствене варијанте једног соја присутне су у желуцима сваког човека, а генетски састав ових популација може се мењати током времена. Ова адаптабилност доприноси и њеној високој заразности \cite{Noto2012-rr}.

Већина инфекција се јавља у детињству, а само мали проценат инфекција напредује до тежих стања. Инфекција овом бактеријом може изазвати различите гастроинтестиналне проблеме, укључујући хронични гастритис, чир на дванаестопалачном цреву, па чак и рак. Хеликобактер пилори је веома заразна бактерија \cite{2022helicobacter}.

\section{Алгоритми истраживања образаца}

\subsection{TF-IDF}

TF-IDF (Term Frequency Inverse Document Frequency) алгоритам
 је
статистички метод који се користи за процену важности речи
за документ или категорију у скупу датотека или корпусу. Главна
идеја је да ако се нека реч или фраза често појављује у
чланку, а ретко се налази у другим члaнцима, сматра се
да реч или фраза има добру способност разликовања класе
и погодна је за класификацију. То је најчешће коришћена
функција за израчунавање тежине речи у тренутном
векторском моделу простора. Углавном се састоји од
два дела, а то су учесталост речи и инверзна учесталост
текста. Учесталост речи се односи на број појављивања
дате речи у датотеци. Инверзна учесталост датотеке представља
меру опште важности речи. Инверзна учесталост речи се дели
укупним бројем докумената, који се дели бројем
докумената који садрже тај термин, а затим се логаритмује
резултат количника. Формуле за учесталост речи (TF) и
инверзну учесталост текста (IDF) су следеће:
$$\mathrm{tf}_{i,j} = \frac{n_{i,j}}{\sum_{k} n_{k,j}}$$

$n_{i,j}$  је број појављивања речи ti у датотеци $d_{j}$,
$\sum_{k} n_{k,j}$ је збир појављивања свих речи у датотеци
$d_{j}$.

$$\mathrm{idf}_i = \log \left(\frac{|D|}{|\{j : t_i \in d_j\}|}\right)

$|D|$ је укупан број датотека у корпусу, $|\{j : t_i \in d_j\}|$
је број докумената који садрже реч $t_{i}$. Ако
речи нема у корпусу, то ће довести до деобе са нулом. Зато се
уопштено користи 1+$|\{j : t_i \in d_j\}|$.

$$\mathrm{tfidf}_{i,j} = \frac{\mathrm{tf}_{i,j} \times \mathrm{idf}_{i}}{\sqrt{\sum_{t_i \in d_j} \left[\mathrm{tf}_{i,j} \times \mathrm{idf}_i\right]^2}}$$


$\mathrm{tfidf}_{i,j}$ је тежина речи ${t}_{i}$. Може се видети да висока
учесталост речи у одређеној датотеци и ниска учесталост датотеке
речи у целом скупу датотека могу генерисати високу TF-IDF вредност. \cite{Liu2018-oa}

\newpage
\bibliography{seminarski}

\end{document}
